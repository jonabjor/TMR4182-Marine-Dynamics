\documentclass[11pt, a4paper, english, hidelinks]{NTNUoving}
\usepackage[utf8]{inputenc}
\usepackage[T1]{fontenc}
\usepackage{tikz}
\usepackage{pgfplots}
\usepackage{graphicx}
\usepackage[header]{appendix}
\usepackage{siunitx}
\usepackage{hyperref}
\usepackage[nottoc]{tocbibind}
% Uncomment this when compiling last time
%\usepackage{minted}
\usepackage{booktabs}
\usepackage{tabularx}
\usepackage{graphicx}
\usepackage{subcaption}


\definecolor{bg}{rgb}{0.95,0.95,0.95}

\usepackage{float}
\usepackage{amsmath}
\usepackage{float}
\usepackage{caption}
\ovingnr{3}
\ovingstekst{Jonas Bjørlo \\ Exercise}
\semester{Marine Dynamics}
\fag{TMR4182}
\institutt{Department of Marine Technology}

\begin{document}
\tableofcontents
\section{Implementation of algorithms}
The algorithms to be implemented are the constant average acceleration algorithm and the Runge-Kutta 4th order algorithm. The system to be solved is a SDOF system where the equation of motion is given by:
\begin{equation}
	\label{eq:eqnofmotion}
	m\ddot{u} + c\dot{u} + ku = P(t)
\end{equation}
\subsection{Constant average acceleration method}
Upon initialization of the constant average method (CAA), we calculate the value of $\ddot{u}_0$ from the equation of motion
\begin{equation}
	\label{eq:eqnofmotionrewrite}
	\ddot{u}_0 = \frac{1}{m}(P_0 - c\dot{u}_0 - ku_0)
\end{equation}
where $\dot{u}_0$ and $u_0$ are the initial conditions. We can now find the displacement at time $t_{i+1}$ from the equation:
\begin{equation}
	\label{eq:caadisplacement}
	\left(\frac{4}{h^2} m + \frac{2}{h}c + k \right)u_{i+1} = P_{i+1} + m\ddot{u}_i + \left(\frac{4}{h}m + c \right) \dot{u}_i + \left(\frac{4}{h^2}m + \frac{2}{h}c\right)u_i
\end{equation}
With the displacement at time $t_{i+1}$, the acceleration $\ddot{u}_{i+1}$ is given by
\begin{equation}
	\label{eq:caaacc}
	\ddot{u}_{i+1} = \frac{4}{h^2}(u_{i+1} - u_i - \dot{u}_ih)-\ddot{u}_i
\end{equation} 
The last step in the process is to calculate $\dot{u}_{i+1}$ given by
\begin{equation}
	\label{eq:caavelocity}
	\dot{u}_{i+1} = \dot{u}_i + \frac{h}{2}(\ddot{u}_i + \ddot{u}_{i+1})
\end{equation}
The algorithm then jumps to the next timestep, and calculates step \eqref{eq:caadisplacement}, \eqref{eq:caaacc} and \eqref{eq:caavelocity} for the values at the timestep. The implementation in Python is given in Appendix \ref{app:pythoncaa}.
\subsection{4th order Runge-Kutta algorithm}
To implement the Runge-Kutta algorithm we need to write the equation of motion as two first order ODE's:
\begin{align}
	\label{eq:firstorderode}
	\dot{u} &= \dot{x}_1 = x_2 \\
	\ddot{u} &= \dot{x}_2 = \frac{1}{m}(P-cx_2-kx_1)
\end{align}
This will be our function $f$ that computes the first derivative. We can write it in the form
\begin{equation}
	f(x,t) = \begin{cases}
	x_2 \\
	\frac{1}{m}(P-cx_2-kx_1)
	\end{cases}
\end{equation}
where
$$
x = \begin{bmatrix}
x_1 \\ x_2
\end{bmatrix}
$$
We start by calculating the derivative at the current point
\begin{equation}
	\label{eq:k0}
	K_0 = \begin{bmatrix}
	K_{0\dot{x}_1} \\
	K_{0\dot{x}_2}
	\end{bmatrix} = f(t_i,x_i)
\end{equation}
Next we jump to the intermediate point $t_{i+0.5}$, and estimate the slope by computing $K_1$
\begin{equation}
	\label{eq:k1}
	K_1 = \begin{bmatrix}
	K_{1\dot{x}_1} \\
	K_{1\dot{x}_2}
	\end{bmatrix} = f(t_{i+0.5}, x_i + \frac{h}{2}K_0)
\end{equation}
Using $K_1$ we can get a better estimate of the derivative at $t_{i+0.5}$:
\begin{equation}
	\label{eq:k2}
	K_2 = \begin{bmatrix}
	K_{2\dot{x}_1} \\
	K_{2\dot{x}_2}
	\end{bmatrix} = f(t_{i+0.5}, x_i + \frac{h}{2}K_1)
\end{equation}
Using $K_2$ we find the derivative at $t_{i+1}$:
\begin{equation}
	\label{eq:k3}
	K_3 = \begin{bmatrix}
	K_{3\dot{x}_1} \\
	K_{3\dot{x}_2}
	\end{bmatrix} = f(t_{i+1},x_i+hK_2)
\end{equation}
With the slopes, we can then compute $x$ using:
\begin{equation}
	\label{eq:RK4weight}
	x_{i+1} = x_i + \frac{h}{6}(K_0 + 2K_1+2K_2+K_3)
\end{equation}
After this step, the process is repeated for the next timestep. 
For the implementation we need to keep in mind that we in fact are computing $K_0$, $K_1$, $K_2$ and $K_3$ for both $\dot{x}_1$ and $\dot{x}_2$. The implementation in Python is given in Appendix \ref{app:pythonrk4}.
\section{Application}
\subsection{Simple harmonic load}
\subsubsection*{Comparison between steady state response and the DLF}
The first step is to choose input values and plot the results from the numerical integrations. Input values for $m$, $c$, $k$, $P_0$ and the 5 values for $\omega$ is given in Table \ref{tab:givenvalue} and \ref{tab:omegarange}. The values of $\omega$ is carefully chosen to give a frequency ratio $\beta$ around $1$.
\begin{table}
	\parbox{.45\linewidth}{
		\centering
		\begin{tabular}{ccc}
			\toprule
			Parameter & Value\\
			\midrule
			$m$ [kg] & 3 \\
			\midrule
			$k$ [N/m] & 10 \\
			\midrule
			$c$ [N/s] & 0.5 \\
			\midrule 
			$P_0$ [N] & 1 \\
			\bottomrule
		\end{tabular}
		\caption{Given values}
		\label{tab:givenvalue}
	}
	\hfill
	\parbox{.45\linewidth}{
		\centering
		\begin{tabular}{ccc}
			\toprule
			Parameter & Value\\
			\midrule
			$\omega_1$ [rad/s] & 1.3 \\
			\midrule
			$\omega_2$ [rad/s] & 1.5 \\
			\midrule
			$\omega_3$ [rad/s] & 1.7 \\
			\midrule 
			$\omega_4$ [rad/s] & 2.0 \\
			\midrule
			$\omega_5$ [rad/s] & 2.3 \\
			\bottomrule
		\end{tabular}
		\caption{Range of 5 values of $\omega$}
		\label{tab:omegarange}
	}
\end{table}

The DLF is calculated for all frequencies:
\begin{equation}
	\label{eq:DLF}
	DLF = \frac{1}{\left( (1-\beta)^2 + (2\zeta \beta)^2\right)^{\frac{1}{2}}}
\end{equation}
A plot of the DLF as a function of the frequency ratio for the 5 values of $\omega$ is given in Figure \ref{fig:DLF}. 
\begin{figure}
	\centering
	\includegraphics[scale=0.5]{DLF_plot.PNG}
	\caption{DLF as a function of the frequency ratio}
	\label{fig:DLF}
\end{figure}
The responses computed from the numerical responses are given in Figure \ref{fig:SSresponse}. 
\begin{figure}
	\begin{subfigure}{.5\textwidth}
		\centering
		\includegraphics[width=.9\linewidth]{w1_wo_P.PNG}
		\caption{$\omega = 1.3$}
		\label{fig:sfig1}
	\end{subfigure}%
	\begin{subfigure}{.5\textwidth}
		\centering
		\includegraphics[width=.9\linewidth]{w2_wo_P.PNG}
		\caption{$\omega = 1.5$}
		\label{fig:sfig2}
	\end{subfigure}
	\begin{subfigure}{.5\textwidth}
		\centering
		\includegraphics[width=.9\linewidth]{w3_wo_P.PNG}
		\caption{$\omega = 1.7$}
		\label{fig:sfig3}
	\end{subfigure}
	\begin{subfigure}{.5\textwidth}
		\centering
		\includegraphics[width=.9\linewidth]{w4_wo_P.PNG}
		\caption{$\omega = 2.0$}
		\label{fig:sfig4}
	\end{subfigure}
	\begin{subfigure}{\textwidth}
		\centering
		\includegraphics[width=.5\linewidth]{w5_wo_P.PNG}
		\caption{$\omega = 2.3$}
		\label{fig:sfig5}
	\end{subfigure}
	\caption{Plots from the numerical methods, for different $\omega$}
	\label{fig:SSresponse}
\end{figure}
Comparing the steady-state response from the graphs and the computed DLF value, we see that a higher value of the DLF means that the amplitude of the steady state response is greater. For $\omega = 1.7$ we have the highest DLF and also the highest amplitude of the steady state response. The phase for each response is computed using:
\begin{equation}
	\label{eq:phase}
	\phi = \arctan{\frac{-2\zeta\beta}{1-\beta^2}}
\end{equation}

To get the right angle in Python, we need to know in which quadrant the solution is. Using the numpy function atan2, we get the solution from the right quadrant. 

From the results we can see that for increasing frequencies, the phase is increasing towards $\phi = -\pi$. Increasing the frequency of the load, means that we will have a bigger frequency ratio. From equation \eqref{eq:phase} that implies that the phase angle will increase. 

The response $u(t)$ is plotted against the load $P(t)$ in Appendix \ref{app:SSload}. These plots show that increasing values of $\omega$ gives increasing phase angles $\phi$.

The Python code for this task is given in Appendix \ref{app:pythondlf}. 
\subsubsection*{Effect of the initial conditions and damping level}
Altering the initial conditions $\dot{u}_0$ and $u_0$, and the damping coefficient $c$ was done in Python. The chosen values is shown in the title of the graphs. The results are given in Figure \ref{fig:transientdamping}, FIG2, FIG2.

\begin{figure}
	\begin{subfigure}{.5\textwidth}
		\centering
		\includegraphics[width=.9\linewidth]{small_damp.PNG}
		\caption{Low damping, $c = 0.5$}
		\label{fig:sfig22}
	\end{subfigure}%
	\begin{subfigure}{.5\textwidth}
		\centering
		\includegraphics[width=.9\linewidth]{big_damp.PNG}
		\caption{Higher damping, $c = 7.0$}
		\label{fig:sfig23}
	\end{subfigure}
	\caption{Response $u(t)$ for two different damping coefficients}
	\label{fig:transientdamping}
\end{figure}


From figure \ref{fig:transientdamping} we can see that the settling time is higher for the system with the lowest damping coefficient. For the system with higher damping, we can see that the transient response almost instantly decays. Having a higher damping coefficient therefore means that we will reach the steady state response quicker. The amplitude of the transient response is also affected by the damping. As the graphs shows, higher damping means lower amplitude on the transient response.




\subsection{Changing load frequency and amplitude}
\subsection{Short impulsive load}

\newpage
\appendix
\appendixpage
\addappheadtotoc
\section{Python implementation of algorithms}
\subsection{Constant average acceleration method}
\label{app:pythoncaa}
% Add Python code from ../const_avg_acc.py
%\inputminted[linenos,bgcolor=bg]{python}{../const_avg_acc.py}

\subsection{Runge-Kutta 4th order algorithm}
\label{app:pythonrk4}
% Add Python code from ../runge_kutta4.py
%\inputminted[linenos,bgcolor=bg,breakbefore=P]{python}{../runge_kutta4.py}
\section{Python implementation simple harmonic load}
\subsection{Steady state response and DLF comparison}
\label{app:pythondlf}

\section{Plots}
\subsection{Steady state response and load plots}
\label{app:SSload}

\begin{figure}
	\begin{subfigure}{.5\textwidth}
		\centering
		\includegraphics[width=.9\linewidth]{w1.PNG}
		\caption{$\omega = 1.3$}
		\label{fig:sfig11}
	\end{subfigure}%
	\begin{subfigure}{.5\textwidth}
		\centering
		\includegraphics[width=.9\linewidth]{w2.PNG}
		\caption{$\omega = 1.5$}
		\label{fig:sfig21}
	\end{subfigure}
	\begin{subfigure}{.5\textwidth}
		\centering
		\includegraphics[width=.9\linewidth]{w3.PNG}
		\caption{$\omega = 1.7$}
		\label{fig:sfig31}
	\end{subfigure}
	\begin{subfigure}{.5\textwidth}
		\centering
		\includegraphics[width=.9\linewidth]{w4.PNG}
		\caption{$\omega = 2.0$}
		\label{fig:sfig41}
	\end{subfigure}
	\begin{subfigure}{\textwidth}
		\centering
		\includegraphics[width=.5\linewidth]{w5.PNG}
		\caption{$\omega = 2.3$}
		\label{fig:sfig51}
	\end{subfigure}
	\caption{Load $P(t)$ and response $u(t)$, for different $\omega$}
	\label{fig:SSresponseP}
\end{figure}
\end{document}