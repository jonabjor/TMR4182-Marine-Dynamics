\documentclass[11pt, a4paper, english]{NTNUoving}
\usepackage[utf8]{inputenc}
\usepackage[T1]{fontenc}
\usepackage{tikz}
\usepackage{pgfplots}
\usepackage{graphicx}
\usepackage{siunitx}
\usepackage[nottoc]{tocbibind}

\usepackage{float}
\usepackage{amsmath}
\usepackage{float}
\usepackage{caption}
\ovingnr{3}
\ovingstekst{Jonas Bjørlo \\ Exercise}
\semester{Marine Dynamics}
\fag{TMR4182}
\institutt{Department of Marine Technology}

\begin{document}
\section{Implementation of algorithms}
\subsection{Constant average acceleration method}
Upon initialization of the constant average method (CAA), we calculate the value of $\ddot{u}_0$ from the equation of motion
\begin{equation}
	\label{eq:eqnofmotion}
	\ddot{u}_0 = \frac{1}{m}(P_0 - c\dot{u}_0 - ku_0)
\end{equation}
where $\dot{u}_0$ and $u_0$ are the initial conditions. We can now find the displacement at time $t_{i+1}$ from the equation:
\begin{equation}
	\label{eq:caadisplacement}
	\left(\frac{4}{h^2} m + \frac{2}{h}c + k \right)u_{i+1} = P_{i+1} + m\ddot{u}_i + \left(\frac{4}{h}m + c \right) \dot{u}_i + \left(\frac{4}{h^2}m + \frac{2}{h}c\right)u_i
\end{equation}
With the displacement at time $t_{i+1}$, the acceleration $\ddot{u}_{i+1}$ is given by
\begin{equation}
	\label{eq:caaacc}
	\ddot{u}_{i+1} = \frac{4}{h^2}(u_{i+1} - u_i - \dot{u}_ih)-\ddot{u}_i
\end{equation} 
The last step in the process is to calculate $\dot{u}_{i+1}$ given by
\begin{equation}
	\label{eq:caavelocity}
	\dot{u}_{i+1} = \dot{u}_i + \frac{h}{2}(\ddot{u}_i + \ddot{u}_{i+1})
\end{equation}
The algorithm then jumps to the next timestep, and calculates step \eqref{eq:caadisplacement}, \eqref{eq:caaacc} and \eqref{eq:caavelocity} for each iteration. The implementation is given in Appendix I
\subsection{4th order Runge-Kutta algorithm}
To implement the Runge-Kutta algorithm we need to write the equation of motion as two first order ODE's:
\begin{align}
	\label{eq:firstorderode}
	\dot{u} &= \dot{x}_1 = x_2 \\
	\dot{u} &= \dot{x}_2 = \frac{1}{m}(P-cx_2-kx_1)
\end{align}
This will be our function $f$ that computes the first derivative. We can write it in the form
\begin{equation}
	f(x,t) = \begin{cases}
	x_2 \\
	\frac{1}{m}(P-cx_2-kx_1)
	\end{cases}
\end{equation}
We start by calculating the derivative at the current point
\begin{equation}
	K_0 = f(t_i,x_i)
\end{equation}
\section{Application}
\subsection{Simple harmonic load}
\subsection{Changing load frequency and amplitude}
\subsection{Short impulsive load}
\end{document}